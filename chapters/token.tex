\chapter{The Legacy Token} % (fold)
\label{cha:the_legacy_token}

% - Contract execution
% - commercial advantages
% - Peer-to-peer services build on top of the Legacy platform 
% - universal access, reduced transaction fees

The Legacy token, called LEG, will be created during a token sale event organized to fund the project. The token sale process will follow standard modalities established in the blockchain community. Once the token sale period is finished, no additional tokens will be generated. A maximum of 100.000.000 LEG can be created. 
Table \ref{table:ico_summary} provides some details regarding the token and its supply distribution. Additional details regarding the token sale process and how to participate in it will be provided in Legacy's official website \cite{Legacy}.

\begin{table}[h]
	\begin{center}		
		{\renewcommand{\arraystretch}{1.3}			
			\begin{tabular}{| c | p{5cm} | p{3.5cm}  |}	
		    \hline	
		    %\textbf{RA scheme} 	& \textbf{Peak  $T_{max}$ [b/s/Hz]}	& 	\textbf{$T$ at target $P_e\leq10^{-3}$ [b/s/Hz]}  &  \textbf{Observations}  \\ \hline    
		    	Total Supply		&  100 000 000 [LEG]  \\ \hline
		    	Auction Model       &  Ascending-price auction \\ \hline														
				Percentage of supply available for investors & 75\% \\ \hline
				Percentage of supply for Legacy founders     &  6\% \\ \hline
				Percentage of supply for Legacy Organization & 15\% \\ \hline
				Percentage of supply for advisors, partners and consultants & 4\% \\
			\hline	
			\end{tabular}				
		}
	\caption{Token sale summary.}
	\label{table:ico_summary}		
	\end{center}
\end{table}



The Legacy token serves several different purposes. 

% First, it is used internally to execute the smart contracts residing in the blockchain.
% After paying subscription fees, which can be done through different currencies, user's funds are converted into LEGs and are locked into a custom user wallet for the entire duration of the service (\textit{i.e.} until the user dies). Service fees are charged periodically from the user wallet.

First, they allow to create a shared economy on top of Legacy's platform. In this way, once Legacy is ready to handle a large variety of assets, experts (\textit{e.g.}, lawyers, accountants) in the community may provide technical assistance to users with complex holdings or in case specific legal requirements must be met according to local regulations related to property disposition through wills. In this context, the legacy token can be used to enable peer-to-peer payments in the platform.

Second, the tokens can be used to encourage constant development of the platform. Users may propose novel functionalities (for instant, a specific PoL plug in) which can be implemented by developers in the community. To further encourage platform development, a special reserve of tokens can be exclusively use for this purpose.

Third, LEG tokens can be used by users to gain access to commercial advantages. Paying for the service directly in LEG gives access to reduced service costs and other type of commercial incentives. This is a standard strategy to encourage token demand and can also allows to implement fiscal policies regarding the token economics \cite{Sehra2017}.

Finally, LEGs allow for profit sharing. As a means for encouraging participation in the token sale, future token holders may benefit from Legacy's profits by depositing LEGs in a custom wallet or contract.

A simplified diagram showing how the token is employed in the application, including the profit sharing mechanism, is presented in Appendix \ref{cha:economic_flow_model} (note that this diagram does not include the use of the token for peer-to-peer transactions, which is a feature considered for future releases).

% \section{Reinforcing Price Stability} % (fold)
% \label{sec:reinforcing_price_stability}
% As detailed in Legacy's business model (available in \cite{Legacy}), users will have the alternative of paying subscription fees on a one-time basis (though they may make additional payments to get exclusive services or premium features if needed), which gives access to the service for life. 
% While this approach provides commercial advantages (\textit{e.g.}, users get a clear idea of the overall service cost and do not risk to see service interruptions due to lack of funds on their accounts), it also involves some technical challenges. In particular, since the service duration for each user is a random variable, the actual overall service cost is a random variable as well.
% Another problem is that operational costs (\textit{i.e.}, blockchain transaction fees or storage costs) may increase in time in unpredictable ways, which adds uncertainty to the overall service cost.
% As a consequence, it is desirable to reinforce LEG's price stability and encourage its valuation in time. Several measures can be adopted for this. The first and most obvious is to incentive demand for the application and for payments using the LEG token.  

% section reinforcing_price_stability (end)

% chapter the_legacy_token (end)