\chapter{The Legacy Token} % (fold)
\label{cha:the_legacy_token}

The Legacy token, called LEG, will be created during a token sale event organized to fund the project. The token sale process will follow standard modalities established in the blockchain community. Once the token sale period is finished, no additional tokens will be generated. A maximum of 100.000.000 LEG can be created. 
Table \ref{table:ico_summary} provides some preliminary\footnote{The details regarding token supply distribution are subject to further changes as the exact crowdsale modality is not yet defined.} details regarding the token and its supply distribution. Additional details regarding the token sale process and how to participate in it will be provided in Legacy's official website \cite{Legacy}.

\begin{table}[h]
	\begin{center}		
		{\renewcommand{\arraystretch}{1.3}			
			\begin{tabular}{| l | p{5cm} | p{3.5cm}  |}	
		    \hline			    
		    	Total Supply		&  100 000 000 [LEG]  \\ \hline % TO-DO
		    	Auction Model       &  fixed-price auction\tablefootnote{To be confirmed.} \\ \hline														
				Percentage of supply available for crowdsale & 60\% \\ \hline
				Percentage of supply for Legacy founders     &  6\% \\ \hline
				Percentage of supply for Legacy Organization & 15\% \\ \hline
				Percentage of supply for advisors, partners and consultants & 12\% \\ \hline
				Percentage of supply for Legacy Foundation & 7\% \\	\hline	
			\end{tabular}				
		}
	\caption{Token sale summary.}
	\label{table:ico_summary}		
	\end{center}
\end{table}



The Legacy token serves three main purposes:

\begin{enumerate}
	\item It allows to create a shared economy on top of Legacy's platform. In this way, once Legacy is able to handle digital assets holding monetary value, experts and professionals such as lawyers, accountants or notaries may provide technical assistance to users managing complex holdings or in case specific legal requirements must be met according to local regulations related to property disposition through wills. In this context, the legacy token can be used to enable peer-to-peer transfer of value on the platform.
	\item In line with the previous idea, tokens can be used to implement a reward and incentive system allowing to encourage continuous platform development. Users may propose novel functionalities (for instant, a specific PoL plugin or a novel storage system) which can be implemented by developers in the community. Once a novel functionality is added to the platform, a fraction of the service fees are sent to its authors.
	This idea is further discussed in Section \ref{sub:a_reward_system_to_reinforce_long_term_service_availability}.
	\item Finally, LEG tokens can be used by users to gain access to commercial advantages. Paying for the service directly in LEG may give access to reduced service costs and other types of commercial incentives. This is a standard strategy to encourage token demand and also allows to implement fiscal policies regarding the token economics \cite{Sehra2017}.

\end{enumerate}

% TO-DO: add diagram
A simplified diagram showing how the token is employed in the application is presented in Appendix \ref{cha:economic_flow_model} (note that this simplified model does not include the use of the token for peer-to-peer transfer of value).

% \section{Legacy's Economic Model} % (fold)
% \label{sec:legacy_s_economic_model}
% TO-DO
% section legacy_s_economic_model (end)

% chapter the_legacy_token (end)